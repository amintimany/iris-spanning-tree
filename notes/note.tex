\documentclass[nocopyrightspace]{sigplanconf}

\usepackage{amsmath}
\usepackage{amsthm}
\usepackage{amssymb}

\usepackage{listings}
\usepackage{lstcoq}

\usepackage[dvipsnames]{xcolor}
\usepackage[T1]{fontenc}
\usepackage[scaled=0.85]{beramono}

\usepackage{todonotes}

\newtheorem{lemma}{Lemma}

\lstdefinelanguage[]{MyML}[]{ML}
{% general command to set parameter(s)
basicstyle=\small\ttfamily, % print whole listing small
keywordstyle=\color{blue}\bfseries,
morekeywords=[2]{CAS, true, false, null},
keywordstyle=[2]\bfseries,
identifierstyle=, % nothing happens
commentstyle=, % white comments
stringstyle=\ttfamily, % typewriter type for strings
showstringspaces=false, % no special string spaces
numbers=left,
numberstyle=\tiny,
stepnumber=1,
numbersep=5pt,
escapeinside={(*@}{@*)}
}
\def\MyMLe{\lstinline[language=MyML, basicstyle=\small\ttfamily]}

\newcommand{\defeq}{\overset{\Delta}{=}}
\newcommand{\path}{\mathit{path}}
\newcommand{\connected}{\mathit{connected}}
\newcommand{\nodes}{\mathit{nodes}}
\newcommand{\front}{\mathit{front}}
\newcommand{\maximal}{\mathit{maximal}}
\newcommand{\Left}{\mathit{left}}
\newcommand{\Right}{\mathit{right}}
\newcommand{\tree}{\mathit{tree}}
\newcommand{\Null}{\mathit{null}}
\newcommand{\strictSG}{\subseteq_{\mathit{strict}}}

\newcommand{\dom}{\mathit{dom}}
\newcommand{\None}{\mathit{None}}
\newcommand{\Some}{\mathit{Some}}

\newcommand{\val}{\mathit{val}}
\newcommand{\loc}{\mathit{loc}}
\newcommand{\option}{\mathit{option}}
\newcommand{\Excl}{\mathit{Excl}}
\newcommand{\excl}{\mathit{excl}}
\newcommand{\finmap}{\overset{\mathit{fin}}{\rightharpoonup}}

%change later to match other iris papers.
\newcommand{\own}{\mathfrak{own}}

\title{Notes on the verification of the concurrent spanning tree algorithm}
\authorinfo{}{}{}

\begin{document}
\maketitle

\begin{abstract}
We give a short description of the verification of the concurrent algorithm for computing
a spanning tree of a graph described below.
\end{abstract}

\begin{figure}
\begin{lstlisting}[mathescape=true, language=MyML]

let try_mark x = CAS(x.mark, false, true)
let unmark_fst x = x.left := null
let unmark_snd x = x.right := null

let rec span x =
  if (x == null) then
    false (*@\label{x-null}@*)
  else
    if try_mark then (*@\label{the-cas}@*)
      let (l, r) =
        (span (x.left) || span (x.right)) in
      if ($\lnot$l) then unmark_fst;
      if ($\lnot$r) then unmark_snd;
      true (*@\label{return-true}@*)
    else
      false (*@\label{cas-fail}@*)
\end{lstlisting}
\caption{The pseudo-code of the spanning tree algorithm in ML style}
\label{fig:code}
\end{figure}

\section{The program and intuition}
The algorithm verified is depicted in Figure~\ref{fig:code}.
It is worth noting that this algorithm does not compute the minimum
spanning tree but rather \emph{a spanning tree}.
In particular, the shape of the computed spanning tree depends on the
scheduling of threads as threads race to mark nodes.
Intuitively, nodes marked by a thread (or its children) form the nodes of
the spanning tree computed by that thread.
Recursively, each thread that manages to mark a node, spawns two
threads to compute spanning trees for its children. 
It then combines those spanning trees to compute a spanning tree with
the node that it has marked as the spanning tree that it computes.
In case a thread fails to mark the node it is called with, it concludes that
that node is or will be processed by some other thread.

In what follows, we first give short intuitive reasoning why the algorithm
does what we claim it does.
Later we discuss the proof in Iris which roughly speaking follows the
same intuitive argument.
We finish by a short comparison with the other work that has formalized
the correctness of the same algorithm.\todo{cite!}

\section{Intuitive resoning}
In this text we assume graphs are doubly branching.
Therefore, we refer to the children of a node as the left or right child (which may not exist for some nodes).
We write $x.\Left = \Null$ and $x.\Right = \Null$ whenever a node $x$ has no left or right child respectively.
Whenever not clear to which graph we are referring we use $g.x.\Left$ and $g.x.\Right$ to refer to the children of a node $x$ in a particular graph $g$.

We write $\path(p, g, x, y)$ to say that $p$ is a path in $g$ from $x$ to $y$.
Here a path $p$ is a sequence
$p = [d_1, \dots, d_n ]$ where $d_1,\dots, d_n \in \{\Left, \Right\}$.

We assume that there is always an empty path from a node to itself.
\[
x \in \nodes(g) \Rightarrow \path([], g, x, x)
\]
A graph is called connected from a node $x$ if for any node in the graph there is a path from $x$ to that node.
\[
\connected(g, x) \defeq \forall y \in \nodes(g).~ \exists p.~\path(p, g, x, y)
\]
where $\nodes(g)$ is the set of nodes of graph $g$.
The front of a set of nodes $A$ in a graph is the set of nodes immediately reachable from nodes in $A$.
\[
\front(g, A) \defeq \{x \mid \exists y \in \nodes(g).~x = y.\Left \lor x = y.\Right \}
\]
A graph is maximal if the set of its nodes is a subset of the front of its nodes.
\[
\maximal(g) \defeq \nodes(g) \subseteq \front(g, \nodes(g))
\]
A tree is a graph
\[
\tree(g, x) \defeq \forall y \in \nodes(g), \exists! p.~\path(p, g, x, y)
\]
A graph $g$ is strict subgraph of $g'$ written as $g \strictSG g'$
if the graph $g$ can be obtained from $g'$ by removing some nodes and children.
\[
\begin{array}{ll}
g \strictSG g' \defeq & \forall x \in \nodes(g).\\
& (g.x.\Left = \Null \lor g.x.\Left = g'.x.\Left)~\land \\
& (g.x.\Right = \Null \lor g.x.\Right = g'.x.\Right)
\end{array}
\]
A graph $g$ is a spanning tree of a graph $g'$ if
\[
g \strictSG g' \land \tree(g, x) \land \nodes(g) = \nodes(g')
\]
\begin{lemma}\label{lem:in_front_nodes}
\[
\forall A, \front(g, A) \subseteq A \Rightarrow A = \nodes(g)
\]
\end{lemma}
\subsection{The intuition}
Let's assume \MyMLe{x} is a node of the graph $g$ and the graph $g$ is connected from $x$.
Note that in such a case, any node \MyMLe{y} with a path from \MyMLe{x} to \MyMLe{y} is also a node of graph $g$.
Therefore, in the recursive calls we can assume that whenever we have a call \MyMLe{span y}, \MyMLe{y} is a node of graph $g$.

Intuitively, \MyMLe{span x} returns true if it has managed to mark \MyMLe{x}.
In that case, we know that there is a graph $g'$ such that
$g' \strictSG g$, $\tree(g', x)$, $\maximal(g')$, nodes of $g'$ are all marked and $\front(g, \nodes(g'))$ are marked (either by us through recursive calls or we have seen them marked through recursive calls).

If \MyMLe{span x} returns \MyMLe{false}, either \MyMLe{x} is \MyMLe{null} or we have seen it marked.

Let's confirm this intuition by going through the code.
The call \MyMLe{span x} returns \MyMLe{false} if \MyMLe{x} is \MyMLe{null}
(in line~\ref{x-null}) or if the \MyMLe{CAS} of line~\ref{the-cas} fails (in line~\ref{cas-fail}).
In the latter case, we know that \MyMLe{x.mark} is not \MyMLe{false} and there it
is \MyMLe{true} (i.e., \MyMLe{x} is marked).

Otherwise, if the \MyMLe{CAS} of line~\ref{the-cas} succeeds, the program returns \MyMLe{true} in line~\ref{return-true}.
In this case, we run \MyMLe{span} on the left and right child of \MyMLe{x} concurrently ad get result.
In case \MyMLe{span} returns \MyMLe{false} for any of the children, we remove it.
In what follows, we show that the desired properties hold.

First note that whenever we have a graph that consists of a node \MyMLe{x} and its children such that
 the children of \MyMLe{x} (if any) are maximal trees, the whole graph is a maximal tree.
Furthermore, since the children (if any) are strict subgraphs of the original graph
and for \MyMLe{x} we have only possibly removed children the whole graph
is a strict subgraph of the original graph.
Finally, note that front of the graph consisting of \MyMLe{x} and its children in the original graph is
the front of the children of \MyMLe{x} in the original graph and their fronts.
We know that these are all marked as guaranteed by the recursive calls on the children of \MyMLe{x} (they are either by the recursive calls or have been seen marked by the recursive calls).

At the top-level call to \MyMLe{span x}, we know that (as the pre-condition) \MyMLe{x} is not \MyMLe{null} and the graph we start with is not marked, it is maximal and it is it is connected from \MyMLe{x}.
Therefore, the call to \MyMLe{span x} can't return \MyMLe{false} as it would indicate \MyMLe{x} was \MyMLe{null} or \MyMLe{x} was marked which is contradiction.
Furthermore, we know that any marked node is marked (recursively) through this call.

Hence, at the top-level call to \MyMLe{span x} assuming \MyMLe{x} is a node of graph $g$, we know that there is a graph $g'$ such that
$g' \strictSG g$, $\tree(g', x)$, $\maximal(g')$, nodes of $g'$ are \emph{exactly those nodes that are marked} and $\front(g, \nodes(g'))$ are all marked.
Hence, we have $\front(g, \nodes(g')) \subseteq \nodes(g')$.
And by Lemma~\ref{lem:in_front_nodes} we have $\nodes(g) = \nodes(g')$.
Therefore, by definition of a spanning tree above we have $g'$ is a spanning tree
of $g$.

\section{Proof in Iris}
In Iris, we formalize graphs as finite maps from memory locations $\mathit{loc}$ to pairs of memory $\option~\loc$.
\begin{Coq}
Definition graph := gmap loc (option loc * option loc).
\end{Coq}

The main theorem proven in Irisi is \Coqe{wp_span} below:
\begin{Coq}
Lemma wp_span g markings (x : val) (l : loc) :
    l $\in$ dom (gset _) g -> maximal g -> connected g l ->
    heap_ctx $\star$
    ([$\star$ map] l $\mapsto$ v $\in$ g,
       $\exists$ (m : loc), markings !! l = Some m $\star$
          l $\mapsto$ ($\#$m, children_to_val v) $\star$ m $\mapsto$ $\#$false) $\vdash$
    WP span (SOME $\#$l)
    {{ _, $\exists$ g',
            ([$\star$ map] l $\mapsto$ v $\in$ g',
              $\exists$ m : loc, markings !! l = Some m $\star$
                l $\mapsto$ ($\#$m, children_to_val v) $\star$ m $\mapsto$ $\#$true)
           $\star$ dom (gset _) g = dom (gset _) g'
           $\star$ $\scriptscriptstyle\blacksquare$ strict_subgraph g g' $\star$ $\scriptscriptstyle\blacksquare$ tree g' l }}.
\end{Coq}
Here $\star$ is the separating conjunction. The symbol $\scriptscriptstyle\blacksquare$ is notation for the modality that
embeds pure propositions (those of type \Coqe|Prop| in Coq) into
the separation logic language of Iris (the equality propositions are pure but they are treated specially in parsing so that this modality is applied without any syntactical cue).
The Iris proposition \Coqe|heap_ctx| is an invariant that relates the heap
assertions, e.g., the points-to assertion $\mapsto$, the physical memory.
The notation \Coqe|[$\star$ map] x $
\mapsto$ v $\in$ g, P | for a finite map \Coqe|g| where \Coqe|x| and \Coqe|v| are free in \Coqe|P| is a Coq notation for $
\bigstar_{x \in A} P(x, g(x))$.
The symbol $\#$ is a Coq notation. It is a part of the programming
language defined to write programs in Iris
\footnote{Iris is language agnostic and different languages can be formalized on top of it. Here we use the one that is defined as part of Iris itself.}.
It internalizes literals like locations and booleans from Coq types to the literal terms in the Iris language.

In this lemma \Coqe|markings| is the mapping that maps each node (location) to a location that contains the marking for that node.
It is necessary to add this mapping as a parameter throughout the
proof to make sure that no part of the program changes the
\emph{the address of} the marking for nodes.
This is crucial as the correctness of the program depends on the
race to mark the node and the fact that as soon as a thread sees
a node marked it stops assuming that another thread has processed or is
processing that node.

\subsection{The recursive contract for \Coqe|wp_span|}
The lemma \Coqe|wp_span| above states that given a location $l$ in a graph $g$
such that $g$ is maximal and connected from $l$, we have the the Hoare triple:
given that we own the graph $g$ in the heap, after running \Coqe|wp_span| we have
that there is a graph $g'$ such that we have $g'$ in the heap and $g$ is a strict sub graph of $g$, $g'$ has the same set of nodes as $g$ and $g'$ is a tree with root $l$.

This lemma in turn is proven using the consequence rule of Hoare logic 
by appealing to a recursive contract for the \Coqe|wp_span| which is
specified in the following lemma.

\begin{Coq}
Lemma rec_wp_span g markings k q (x : val) :
    maximal g ->
    (x = NONEV $\lor$ $\exists$ l : loc,
        x = SOMEV $\#$l $\land$ l $\in$ dom (gset _) g) ->
    (heap_ctx $\star$ graph_ctx $\kappa$ g markings $\star$ own_graph q $\emptyset$ $\star$ cinv_own $\kappa$ k)
      $\vdash$
      WP (span x)
      {{ v, ((v = $\#$ true $\star$
             ($\exists$ l : loc, x = SOMEV $\#$l $\star$
               ($\exists$ (G : Gmon) mr (tr : tree (Gmon_graph G) l),
                  own_graph q G $\star$ $\scriptscriptstyle\blacksquare$ (l $\in$ dom (gset _) G) $\star$
                  $\scriptscriptstyle\blacksquare$ maximal (Gmon_graph G) $\star$
                  $\scriptscriptstyle\blacksquare$ (front g (dom (gset _) G) mr) $\star$
                   ([$\star$ set] l $\in$ mr , is_marked l)) $\star$ is_marked l)) $\lor$
             (v = $\#$ false $\star$
              (x = NONEV $\lor$ ($\exists$ l : loc, x = SOMEV $\#$l $\star$ is_marked l))
               $\star$ own_graph q $\emptyset$))
            $\star$ cinv_own $\kappa$ k }}.
\end{Coq}

This lemma basically states that given a value \Coqe|x| and maximal graph
\Coqe|g| such that \Coqe|x| is either \Coqe|NONEV| (read $\mathit{null}$) or a location in \Coqe|g|, we have the following Hoare triple:
given that we have a partial view of graph \Coqe|own_graph q $\emptyset$| that is we own fraction \Coqe|q| of the empty graph
(since we have not yet marked anything)\footnote {We will explain later why we need this fraction \Coqe|q|.}, after executing \Coqe|span|
we know that one of the following two cases holds.
The return value is true, in which case we own fraction \Coqe|q| of some
graph \Coqe|G|, \Coqe|own_graph q G|, (the spanning tree of the part that we have marked) that
is a tree, is maximal, we know its front is marked and that the node \Coqe|l| we started with is marked.
In the other case, the return value is false and we own the fraction \Coqe|q| of the empty graph (as we have failed to mark any nodes)
and we know that the node \Coqe|l| (if any) that we started with is marked.

Since in this algorithm the graph being worked on is shared among different threads as a whole, we need to enforce a protocol on how it
can be a accessed mutated by different threads.
This is why we talk about the partial view of the graph.
In other words, each thread only has solid knowledge about the shape of
the part of graph that it has marked itself (otherwise, this knowledge could be violated by actions of other threads).

Iris uses partial commutative monoids (here, monoid for short ) to represent its ghost states.
In Iris, one can compose monoids to build larger monoids.
The monoids are shared between different threads using the rule
\[
\own(a \cdot b) \dashv\vdash \own(a) \star \own(b)
\]
where $\own$ is the ownership of the ghost state, $\cdot$ is the monoid
operation and $\dashv\vdash$ is equivalence of Iris propositions.
In the following we explain monoids that are used to build the moinds
that we use on a call by need basis.

In order to represent graphs in the ghost state (\Coqe|g| and \Coqe|G| in the lamma above) we define the following monoids.
\[
\mathit{Gmon} \defeq \loc \finmap \option(\Excl(\val \times \val))
\]
where $\finmap$, $\option$ and $\Excl$ are monoid formers and $\val$ is the set of values of the language.
See the appendix for their definition.
Here, we just say that $\finmap$ is forms a partial map.
This definition is very close to our definition of graph.
The only difference is $\option$ and $\Excl$.
The $\option$ monoid former allows for partiality of knowledge,
we might not have any knowledge.
The exclusive monoid former $\Excl$ enforces exclusiveness of the
information.
If we know that a node has specific children, then other threads
can't have any knowledge about that node.
On other hand, knowing the children of a node allows us to mutate its
children. Notice that this is only possible if we have marked the node
ourselves.
These facts can be seen in the contracts for the \MyMLe{try_mark},
\MyMLe{unmark_fst} and \MyMLe{unmark_snd}.
\begin{Coq}
Lemma wp_try_mark g Mrk k q (x : loc) : x $\in$ dom (gset _) g ->
    heap_ctx $\star$ graph_ctx $\kappa$ g Mrk $\star$ own_graph q $\emptyset$ $\star$ cinv_own $\kappa$ k
    $\vdash$ WP (try_mark $\#$x) {{ v,
         (v = $\#$true $\star$ ($\exists$ u, (g !! x) = Some u $\star$ own_graph q (x [$\mapsto$] u))
          $\star$ is_marked x $\star$ cinv_own $\kappa$ k)
           $\lor$ (v = $\#$false $\star$ own_graph q $\emptyset$ $\star$ is_marked x $\star$ cinv_own $\kappa$ k) }}.
\end{Coq}

\begin{Coq}
Lemma wp_unmark_fst g markings k q (x : loc) v w1 w2 :
    (g !! x) = Some v ->
    (heap_ctx $\star$ graph_ctx $\kappa$ g markings $\star$ own_graph q (x [$\mapsto$] (w1, w2))
     $\star$ cinv_own $\kappa$ k) $\vdash$
      WP (unmark_fst $\#$x)
      {{ _, own_graph q (x [$\mapsto$] (None, w2)) $\star$ cinv_own $\kappa$ k }}.
\end{Coq}

\begin{Coq}
Lemma wp_unmark_snd g markings k q (x : loc) v w1 w2 :
    (g !! x) = Some v ->
    (heap_ctx $\star$ graph_ctx $\kappa$ g markings $\star$ own_graph q (x [$\mapsto$] (w1, w2))
     $\star$ cinv_own $\kappa$ k) $\vdash$
      WP (unmark_snd $\#$x)
      {{ _, own_graph q (x [$\mapsto$] (w1, None)) $\star$ cinv_own $\kappa$ k }}.
\end{Coq}

\appendix
\section{Monoids}
The monoid former $\finmap$ is constructs finite maps from a type to a monoid.
The composition operation $\cdot$ is defined as follows:
\[
(f \cdot g) (x) = \left\{
\begin{array}{ll}
f(x) & \text{if } x \in \dom(f)\setminus\dom(g)\\
g(x) & \text{if } x \in \dom(g)\setminus\dom(f)\\
f(x) \cdot g(x) & \text{if } x \in \dom(f)\cap\dom(g)\\
\bot & \text{otherwise}
\end{array}
\right.
\]
The monoid former of $\option(A)$ for a monoid $A$ has as elements 
$\None$ or $\Some(x)$ for $x \in A$ with:
\[
\begin{array}{ll}
\None \cdot \None = \None & \None \cdot \Some(x) = \Some(x) \\
\multicolumn{2}{c}{\Some(x) \cdot \Some(y) = \Some(x \cdot y)}
\end{array}
\]

The monoid former of $\option(A)$ for a monoid $A$ has as elements 
$\None$ or $\Some(x)$ for $x \in A$ with:
\[
\begin{array}{ll}
\None \cdot \None = \None & \None \cdot \Some(x) = \Some(x) \\
\multicolumn{2}{c}{\Some(x) \cdot \Some(y) = \Some(x \cdot y)}
\end{array}
\]

The monoid former of $\excl(A)$ for a \emph{type} $A$ has as elements 
$\excl(x)$ for $x \in A$ with:
\[
\begin{array}{ll}
a \cdot b = \bot
\end{array}
\]

\end{document}